\documentclass[aspectratio=169,11pt]{beamer}

% =============================================================================
% MODERN MINIMALIST BEAMER TEMPLATE
% =============================================================================
% A clean, black-on-white presentation style with sans-serif typography.
% Designed for academic seminars, conference talks, and thesis defenses.
%
% Features:
% - 16:9 aspect ratio (widescreen)
% - Helvetica font family (sans-serif)
% - No navigation symbols or decorations
% - Full-slide image support via TikZ
% - Tight table formatting for wide data
% =============================================================================

% --- Theme: Minimalistic defaults ---
\usetheme{default}
\usecolortheme{default}

% --- Encoding and Fonts ---
\usepackage[T1]{fontenc}
\usepackage[utf8]{inputenc}
\usepackage{helvet}
\renewcommand{\familydefault}{\sfdefault}

% --- Essential Packages ---
\usepackage{graphicx}
\usepackage{booktabs}
\usepackage{tabularx}
\usepackage{array}
\usepackage{ragged2e}
\usepackage{hyperref}
\usepackage{tikz}
\usetikzlibrary{calc}

% --- Graphics Path ---
\graphicspath{{figures/}}

% --- Remove Navigation Symbols ---
\setbeamertemplate{navigation symbols}{}

% --- Colors: Plain Black on White ---
\setbeamercolor{structure}{fg=black}
\setbeamercolor{normal text}{fg=black,bg=white}
\setbeamercolor{frametitle}{fg=black,bg=white}
\setbeamercolor{block title}{fg=black,bg=white}
\setbeamercolor{block body}{fg=black,bg=white}
\setbeamercolor{background canvas}{bg=white}

% --- Remove Headline and Footline ---
% Prevents overlay/clipping issues on image slides
\setbeamertemplate{headline}{}
\setbeamertemplate{footline}{}

% --- Comfortable Text Margins ---
\setbeamersize{text margin left=8mm,text margin right=8mm}

% --- Clean Frame Title with Horizontal Rule ---
\setbeamertemplate{frametitle}{%
  \vspace*{0.55em}%
  {\Large\bfseries\insertframetitle\par}%
  \vspace*{0.20em}\hrule\vspace*{0.55em}%
}

% --- Cleaner Bullet Points ---
\setbeamertemplate{itemize item}{\textbullet}
\setbeamertemplate{itemize subitem}{\textendash}
\setbeamertemplate{enumerate item}{\arabic{enumi}.}

% =============================================================================
% CUSTOM COMMANDS
% =============================================================================

% Full-Slide Image Helper
% Uses a page overlay to avoid Beamer text margins shifting/cropping images.
% Usage: \FullSlideImage{filename}
\newcommand{\FullSlideImage}[1]{%
  \begin{frame}[plain]
    \begin{tikzpicture}[remember picture,overlay]
      \node[anchor=center] at (current page.center) {%
        \includegraphics[width=\paperwidth,height=\paperheight,keepaspectratio]{#1}%
      };
    \end{tikzpicture}
  \end{frame}
}

% Tight Table Defaults
% Keeps wide tables from spilling/cropping
% Usage: \TightTableSetup before your table environment
\newcommand{\TightTableSetup}{%
  \setlength{\tabcolsep}{4pt}%
  \renewcommand{\arraystretch}{1.15}%
}

% =============================================================================
% TITLE PAGE INFORMATION
% =============================================================================

\title[Short Title]{Your Presentation Title}
\subtitle{Subtitle or Event Context}
\author{Your Name}
\institute{Department or Laboratory\\Institution Name}
\date{Event Name\\Month Year}

% =============================================================================
% DOCUMENT START
% =============================================================================

\begin{document}

% --- Title Slide ---
% Option 1: Use an image for full visual control
% \FullSlideImage{title_slide.png}

% Option 2: Use standard title page
\begin{frame}[plain]
  \titlepage
\end{frame}

% =============================================================================
% CONTENT SLIDES
% =============================================================================

% --- Motivation ---
\begin{frame}{Motivation}
\begin{enumerate}
  \item \textbf{Context}\\
  Describe the broader context or problem domain.
  \item \textbf{Challenge}\\
  Identify the specific challenge or gap.
  \item \textbf{Opportunity}\\
  Explain why this is worth addressing now.
  \item \textbf{Approach}\\
  Preview your solution or contribution.
\end{enumerate}
\end{frame}

% --- Background (as image) ---
% \FullSlideImage{background_diagram.png}

% --- Research Question ---
\begin{frame}{Research Question}
\begin{enumerate}
  \item Prior work has shown X
  \item However, Y remains unclear
  \item This limits our understanding of Z
\end{enumerate}

\vspace{0.55em}
\begin{block}{Guiding Question}
Can we develop a method that achieves X while addressing Y?
\end{block}
\end{frame}

% --- Methods ---
\begin{frame}{Methods}
\begin{columns}[T]
  \begin{column}{0.48\textwidth}
    \textbf{Data}
    \begin{itemize}
      \item Source and collection
      \item Sample size and characteristics
      \item Preprocessing steps
    \end{itemize}
  \end{column}
  \begin{column}{0.48\textwidth}
    \textbf{Analysis}
    \begin{itemize}
      \item Analytical approach
      \item Key metrics
      \item Validation strategy
    \end{itemize}
  \end{column}
\end{columns}

\vspace{0.8em}
\footnotesize
Implementation details: Software, versions, computational resources.
\end{frame}

% --- Pipeline Diagram (as image) ---
% \FullSlideImage{pipeline_diagram.png}

% --- Results Table ---
\begin{frame}{Results Summary}
\TightTableSetup
\small
\begin{center}
\begin{tabularx}{0.9\textwidth}{@{}>{\raggedright\arraybackslash}X ccc@{}}
\toprule
\textbf{Method} & \textbf{Metric 1} & \textbf{Metric 2} & \textbf{Metric 3} \\
\midrule
Baseline A & 0.82 & 0.75 & 12.3s \\
Baseline B & 0.85 & 0.78 & 15.1s \\
\textbf{Proposed} & \textbf{0.94} & \textbf{0.91} & 14.7s \\
\bottomrule
\end{tabularx}
\end{center}

\vspace{0.5em}
\begin{itemize}
  \item Key observation: Proposed method achieves X\% improvement
  \item Secondary finding: Comparable computational cost
\end{itemize}
\end{frame}

% --- Results Visualization (as image) ---
% \FullSlideImage{results_chart.png}

% --- Discussion ---
\begin{frame}{Discussion}
\textbf{Key Findings}
\begin{enumerate}
  \item Finding 1 and its significance
  \item Finding 2 and its implications
  \item Finding 3 and connection to prior work
\end{enumerate}

\vspace{0.55em}
\textbf{Limitations}
\begin{itemize}
  \item Limitation 1 and potential mitigation
  \item Limitation 2 and future work direction
\end{itemize}
\end{frame}

% --- Conclusions ---
\begin{frame}{Conclusions}
\begin{enumerate}
  \item \textbf{Main contribution}\\
  Summary of primary finding or method.
  \item \textbf{Key results}\\
  Quantitative summary of outcomes.
  \item \textbf{Future directions}\\
  Next steps and open questions.
\end{enumerate}

\vspace{0.55em}
\begin{block}{Take-home Message}
One sentence capturing the core contribution of this work.
\end{block}
\end{frame}

% --- Thank You / Questions ---
\begin{frame}[plain]
\begin{center}
\vspace{2cm}
{\Large \textbf{Thank You}}

\vspace{1cm}
{\large Questions?}

\vspace{1.5cm}
\begin{tabular}{c}
Your Name \\
\texttt{email@institution.edu} \\
\url{https://yourwebsite.edu}
\end{tabular}

\vspace{1cm}
\footnotesize
Acknowledgments: Funding sources, collaborators, etc.
\end{center}
\end{frame}

% =============================================================================
% BACKUP SLIDES (Optional)
% =============================================================================

\appendix

\begin{frame}{Backup: Additional Analysis}
Additional details for anticipated questions.
\begin{itemize}
  \item Extended results
  \item Sensitivity analyses
  \item Methodological details
\end{itemize}
\end{frame}

\begin{frame}{Backup: Technical Details}
\begin{itemize}
  \item Implementation specifics
  \item Parameter settings
  \item Computational requirements
\end{itemize}
\end{frame}

\end{document}
